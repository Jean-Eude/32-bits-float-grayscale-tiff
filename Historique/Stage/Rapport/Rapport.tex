\documentclass[12pt,a4paper]{report}
\usepackage[utf8]{inputenc}
\usepackage[T1]{fontenc}
\usepackage[french]{babel}
\usepackage{geometry}
\usepackage{graphicx}
\usepackage{lmodern}
\usepackage{setspace}
\usepackage{fancyhdr}
\pagestyle{fancy}

% Corriger ce que contient \leftmark pour les chapitres
\renewcommand{\chaptermark}[1]{%
  \markboth{CHAPITRE \thechapter. \ #1}{}%
}

\fancyhf{} % Réinitialise tout
\fancyhead[L]{\nouppercase{\leftmark}} % Titre du chapitre
\fancyfoot[L]{Evan COMBOT}
\fancyfoot[C]{\thepage}
\fancyfoot[R]{Université de Montpellier}

% Appliquer aussi aux pages de style "plain"
\fancypagestyle{plain}{%
  \fancyhf{}
  \fancyhead[L]{\nouppercase{\leftmark}}
  \fancyfoot[L]{Evan COMBOT}
  \fancyfoot[C]{\thepage}
  \fancyfoot[R]{Université de Montpellier}
}

\usepackage{titling}
\usepackage{mathpazo}
\usepackage{ragged2e}
\usepackage{titlesec}
\titleformat{\chapter}[display]{\normalfont\huge\bfseries}{\chaptername\ \thechapter}{10pt}{\Huge}
\titlespacing*{\chapter}{0pt}{-20pt}{30pt}
\setcounter{tocdepth}{2}

% === Pour gérer les marges58 ===
\geometry{left=2.0cm, right=2.0cm, top=2.0cm, bottom=2.0cm}
\onehalfspacing
\usepackage[]{hyperref}


\begin{document}
\justifying

\begin{titlepage}
\begin{center}

% === LOGOS ALIGNÉS EN LIGNE AVEC LÉGENDES EN GRAS ===
\vspace*{0.5cm}
\begin{minipage}{0.3\textwidth}
    \centering
    \includegraphics[width=0.85\textwidth]{Assets/UM.png}\\[0.2cm]
    {\footnotesize\textbf{Université de Montpellier}}
\end{minipage}
\hfill
\begin{minipage}{0.3\textwidth}
    \centering
    \includegraphics[width=0.85\textwidth]{Assets/Dpt_Info.png}\\[0.2cm]
    {\footnotesize\textbf{Département Informatique -- IMAGINE}}
\end{minipage}
\hfill
\begin{minipage}{0.3\textwidth}
    \centering
    \includegraphics[width=0.85\textwidth]{Assets/CEA.png}\\[0.2cm]
    {\footnotesize\textbf{CEA -- Commissariat à l'énergie atomique et aux énergies alternatives}}
\end{minipage}

\vspace{1.8cm}

% === ÉTABLISSEMENT & DIPLÔME ===
{\Large \textbf{Université de Montpellier}}\\[0.2cm]
{\normalsize Faculté des Sciences}\\[0.3cm]
{\large \textbf{Master 2 Informatique -- Parcours \og IMAGINE \fg{}}}\\[1.5cm]

% === TITRE DU DOCUMENT ===
{\Huge \bfseries Rapport de Stage}\\[0.4cm]
\rule{0.65\linewidth}{0.7pt}\\[1cm]

{\large \textbf{Exploitation d'images de caractérisation de chaîne radiographique}}\\[0.5cm]
\textit{Stage réalisé au sein du \textbf{Commissariat à l'énergie atomique et aux énergies alternatives}}\\[0.3cm]
\textit{du \textbf{3 février 2025} au \textbf{1 août 2025}}\\[2cm]

% === INFOS ÉTUDIANT & ENTREPRISE ===
\begin{flushleft}
\textbf{Réalisé par :} \hfill COMBOT Evan\\[0.15cm]
\textbf{Année universitaire :} \hfill 2024--2025\\[0.15cm]
\textbf{Tuteur en entreprise :} \hfill Nom Prénom\\[0.15cm]
\textbf{Encadrant pédagogique :} \hfill PUECH William\\[0.15cm]
\textbf{Adresse de l'entreprise :} \hfill CEA Gramat, 46500 Gramat, France
\end{flushleft}

\vfill
{\small Date de rendu : 20 juin 2025}

\end{center}
\end{titlepage}

%======================
% Remerciements
%======================
\chapter*{Remerciements}
\addcontentsline{toc}{chapter}{Remerciements}

Je tiens à remercier toutes les personnes qui m'ont accompagné(e) durant ce stage. En particulier, je souhaite exprimer ma gratitude à :
\begin{itemize}
  \item Mon tuteur de stage en entreprise, M. Nom Prénom, pour son accueil, son encadrement et sa disponibilité ;
  \item Mon encadrant pédagogique, M. PUECH William, pour ses conseils et son suivi régulier ;
  \item Toute l'équipe du CEA pour leur soutien et leur bienveillance.
\end{itemize}

Merci également à mes enseignants du Master 2 IMAGINE pour la qualité de la formation dispensée.

\tableofcontents
\pagestyle{fancy}
\newpage

\chapter{Introduction}
... % contenu identique

\chapter{Présentation de l'entreprise}
... % contenu identique

\chapter{Présentation de la mission}
... % à compléter

\chapter{Environnement technique}
... % à compléter

\chapter{Veille technologique}
... % à compléter

\chapter{Missions réalisées}
... % contenu identique avec tâches, technos, outils, etc.

\chapter{Résultats et compétences développées}
... % contenu identique

\chapter{Conclusion}
... % contenu identique

\appendix
\chapter{Annexes}
... % contenu identique

\end{document}
