\documentclass[12pt,a4paper]{report}
\usepackage[utf8]{inputenc}
\usepackage[T1]{fontenc}
\usepackage[french]{babel}
\usepackage{geometry}
\usepackage{graphicx}
\usepackage{lmodern}
\usepackage{setspace}
\usepackage{fancyhdr}
\pagestyle{fancy}

% Corriger ce que contient \leftmark pour les chapitres
\renewcommand{\chaptermark}[1]{%
  \markboth{CHAPITRE \thechapter. \ #1}{}%
}

\fancyhf{} % Réinitialise tout
\fancyhead[L]{\nouppercase{\leftmark}} % Titre du chapitre
\fancyfoot[L]{Evan COMBOT}
\fancyfoot[C]{\thepage}
\fancyfoot[R]{Université de Montpellier}

% Appliquer aussi aux pages de style "plain"
\fancypagestyle{plain}{%
  \fancyhf{}
  \fancyhead[L]{\nouppercase{\leftmark}}
  \fancyfoot[L]{Evan COMBOT}
  \fancyfoot[C]{\thepage}
  \fancyfoot[R]{Université de Montpellier}
}

\usepackage{titling}
\usepackage{mathpazo}
\usepackage{ragged2e}
\usepackage{titlesec}
\titleformat{\chapter}[display]{\normalfont\huge\bfseries}{\chaptername\ \thechapter}{10pt}{\Huge}
\titlespacing*{\chapter}{0pt}{-20pt}{30pt}
\setcounter{tocdepth}{2}

% === Pour gérer les marges58 ===
\geometry{left=2.0cm, right=2.0cm, top=2.0cm, bottom=2.0cm}
\onehalfspacing
\usepackage[]{hyperref}


\begin{document}
\justifying

\begin{titlepage}
\begin{center}

% === LOGOS ALIGNÉS EN LIGNE AVEC LÉGENDES EN GRAS ===
\vspace*{0.5cm}
\begin{minipage}{0.3\textwidth}
    \centering
    \includegraphics[width=0.85\textwidth]{Assets/UM.png}\\[0.2cm]
    {\footnotesize\textbf{Université de Montpellier}}
\end{minipage}
\hfill
\begin{minipage}{0.3\textwidth}
    \centering
    \includegraphics[width=0.85\textwidth]{Assets/Dpt_Info.png}\\[0.2cm]
    {\footnotesize\textbf{Département Informatique -- IMAGINE}}
\end{minipage}
\hfill
\begin{minipage}{0.3\textwidth}
    \centering
    \includegraphics[width=0.85\textwidth]{Assets/CEA.png}\\[0.2cm]
    {\footnotesize\textbf{CEA -- Commissariat à l'énergie atomique et aux énergies alternatives}}
\end{minipage}

\vspace{1.8cm}

% === ÉTABLISSEMENT & DIPLÔME ===
{\Large \textbf{Université de Montpellier}}\\[0.2cm]
{\normalsize Faculté des Sciences}\\[0.3cm]
{\large \textbf{Master 2 Informatique -- Parcours \og IMAGINE \fg{}}}\\[1.5cm]

% === TITRE DU DOCUMENT ===
{\Huge \bfseries Rapport de Stage}\\[0.4cm]
\rule{0.65\linewidth}{0.7pt}\\[1cm]

{\large \textbf{Exploitation d'images de caractérisation de chaîne radiographique}}\\[0.5cm]
\textit{Stage réalisé au sein du \textbf{Commissariat à l'énergie atomique et aux énergies alternatives}}\\[0.3cm]
\textit{du \textbf{3 février 2025} au \textbf{1 août 2025}}\\[2cm]

% === INFOS ÉTUDIANT & ENTREPRISE ===
\begin{flushleft}
\textbf{Réalisé par :} \hfill COMBOT Evan\\[0.15cm]
\textbf{Année universitaire :} \hfill 2024--2025\\[0.15cm]
\textbf{Tuteur en entreprise :} \hfill Nom Prénom\\[0.15cm]
\textbf{Encadrant pédagogique :} \hfill PUECH William\\[0.15cm]
\textbf{Adresse de l'entreprise :} \hfill CEA Gramat, 46500 Gramat, France
\end{flushleft}

\vfill
{\small Date de rendu : 20 juin 2025}

\end{center}
\end{titlepage}

%======================
% Remerciements
%======================
\chapter*{Remerciements}
\addcontentsline{toc}{chapter}{Remerciements}

Je souhaite exprimer ma profonde gratitude à toutes les personnes qui m'ont soutenu(e) et accompagné(e) tout au long de ce stage.

Je remercie tout particulièrement :
\begin{itemize}
  \item M. Nom Prénom, mon tuteur de stage en entreprise, pour son accueil chaleureux, son encadrement attentif et sa grande disponibilité ;
  \item M. William Puech, mon encadrant pédagogique, pour son suivi rigoureux et sa bienveillance;
  \item L’ensemble de l’équipe du CEA, pour leur soutien constant, leur convivialité, l’excellente ambiance de travail qu’ils ont su instaurer, ainsi que pour m’avoir intégré(e) très rapidement au sein de leur équipe.
\end{itemize}

Je tiens également à remercier l’ensemble des enseignants du Master 2 IMAGINE pour la qualité de l’enseignement et les compétences qu’ils m’ont permis d’acquérir.

\vspace{1em}
Merci à toutes et à tous.

\tableofcontents
\pagestyle{fancy}
\newpage

\chapter{Introduction}
\section{Contexte du stage}
\section{Objectifs du rapport}
Ce rapport à pour objectif de présenter le déroulement de mon stage, réalisé au sein du CEA dans le cadre du stage de fin d'études du MAster 2 IMAGINE.

Plus précisément, ce document a pour buts de :
\begin{itemize}
  \item Expliquer le cadre général du stage et les objectifs fixés en début de mission ;
  \item Décrire les différentes tâches réalisées et les méthodologies employées ;
  \item Mettre en évidence les outils et technologies utilisés dans le contexte professionnel ;
  \item Présenter les résultats obtenus et les éventuelles difficultés rencontrées ;
  \item Faire un bilan personnel et professionnel sur les acquis du stage.
\end{itemize}

L'ensemble de ce rapport permet également la mise en pratique des connaissances théoriques acquises tout au long de la formation tout en intégrant les acquis du stage dans le développement de nouvelles compétences.


... % contenu identique

\chapter{Présentation de l'entreprise}
\section{Historique et activités}
\section{Organisation et équipe}
\section{Valeurs et vision}
... % contenu identique

\chapter{Présentation de la mission}
\section{Contexte de la mission}
\section{Enjeux et objectifs}
... % à compléter

\chapter{Environnement technique}
\section{Technologies utilisées}
\section{Outils de développement}
\section{Contraintes techniques}
... % à compléter

\chapter{Veille technologique}
\section{Sujets explorés}
\section{Outils de veille}
\section{Intégration dans la mission}
... % à compléter

\chapter{Missions réalisées}
\section{Tâches effectuées}
\section{Méthodologie}
\section{Difficultés rencontrées}
... % contenu identique avec tâches, technos, outils, etc.

\chapter{Résultats et compétences développées}
\section{Résultats obtenus}
\section{Compétences techniques}
\section{Compétences transversales}
... % contenu identique

\chapter{Conclusion}
\section{Bilan du stage}
\section{Perspectives}
... % contenu identique

\appendix
\chapter{Annexes}


\end{document}
